\documentclass{article}

\usepackage[utf8]{inputenc}

\title{Ausführbare Anforderungen mit Cucumber}
\author{Uwe Kranz}
\date{9. Februar 2015}

\begin{document}

\maketitle
\thispagestyle{empty}

\begin{abstract}
\setlength\parindent{0pt}Eine Wassermelone ist innen rot und außen grün. Sie verbildlicht das Problem von schönigendem Reporting. Eine Gurke hingegen ist durch und durch grün und veranschaulicht Transparenz. Cucumber ist die Übersetzung vom deutschen Wort Gurke ins Englische und bezeichnet ein Werkzeug des Behavior-Driven-Development. Es dient der Spezifikation von Anforderungen an Software und ihrer automatisierten Überprüfung. Einsatzmöglichkeiten finden sich Beispielsweise im Umfeld von Java und .NET sowie bei diversen Web-Anwendungen. Im Workshop schärfen wir Requirements anhand von Beispielszenarien und bauen sie anschließend zu automatisierten Tests aus. Die so gewonnenen Tests treiben dann unsere Implementierung an, für die wir uns in Paaren zum Coding zusammenfinden. Zum Abschluss reflektieren wir darüber, wann die erprobte Vorgehensweise geeignet ist Kommunikation und Transparenz zu fördern.
\end{abstract}

\end{document}
